%\section{Simulation No1}

%%%%%%%%%%%%%%%%%%%%%%%%%%%%%%%%%%%%%%%%%%%%%%%%%%%%%%%%%%%
\begin{figure}[h]
\centerline{\epsfxsize=9cm\epsfbox{s2_orig.ps}
\hskip-1cm\epsfxsize=9cm\epsfbox{s2_pw_orig.ps}}
\centerline{\epsfxsize=9cm\epsfbox{s2_psf.ps}
\hskip-1cm\epsfxsize=9cm\epsfbox{s2_pw_psf.ps}}
\centerline{\epsfxsize=9cm\epsfbox{s2_filt_orig.ps}
\hskip-1cm\epsfxsize=9cm\epsfbox{s2_pw_filt_orig.ps}}
\centerline{\epsfxsize=9cm\epsfbox{s2_noisy_sig.ps}
\hskip-1cm\epsfxsize=9cm\epsfbox{s2_pw_noisy.ps}}
\caption{Simulation \#2. De bas en haut (� gauche): signal original,
r�ponse impulsionnelle du filtre, signal filtr� et signal filtr� et bruit�.
A droite: spectre de puissance correspondant.}
\label{fig:s2-signal}
\end{figure}

%%%%%%%%%%%%%%%%%%%%%%%%%%%%%%%%%%%%%%%%%%%%%%%%%%%%%%%%%%%
\begin{figure}[h]
\centerline{\epsfxsize=10cm\epsfbox{s2_inverse.ps}
\hskip-1cm\epsfxsize=10cm\epsfbox{s2_inverse_filt.ps}}
\centerline{\epsfxsize=10cm\epsfbox{s2_wiener.ps}
\hskip-1cm\epsfxsize=10cm\epsfbox{s2_wiener_filt.ps}}
\caption{Simulation \#2. 
En haut: simple division spectrale et filtre inverse correspondant.
En bas: deconvolution par filtre de Wiener et filtre correspondant.}
\label{fig:s2-wiener}
\end{figure}

%%%%%%%%%%%%%%%%%%%%%%%%%%%%%%%%%%%%%%%%%%%%%%%%%%%%%%%%%%%
\begin{figure}[h]
\centerline{\epsfxsize=15cm\epsfbox{s2_tikho_01.ps}}
\caption{Simulation \#2. 
D�convolution avec r�gularisation de Tikhonov 
avec $\alpha$=0.1 (tol�rance = $10^{-6}$).}
\label{fig:s2-tikhonov}
\end{figure}

%%%%%%%%%%%%%%%%%%%%%%%%%%%%%%%%%%%%%%%%%%%%%%%%%%%%%%%%%%%
\begin{figure}[h]
\centerline{\epsfxsize=15cm\epsfbox{s2_ggauss_01.ps}}
\caption{Simulation \#2. 
D�convolution avec r�gularisation de Gauss g�n�ralis� 
avec p=1.1 avec $\alpha$=0.1 (tol�rance = $10^{-6}$).}
\label{fig:s2-ggauss}
\end{figure}

%%%%%%%%%%%%%%%%%%%%%%%%%%%%%%%%%%%%%%%%%%%%%%%%%%%%%%%%%%%
\begin{figure}[h]
\centerline{\epsfxsize=15cm\epsfbox{s2_sqrt_01_001.ps}}
\caption{Simulation \#2. 
D�convolution avec r�gularisation convexe $\sqrt{s^2+x^2}$
avec $s=0.001$ ($\alpha$=0.1, tol�rance = $10^{-6}$).}
\label{fig:s2-sqrt001}
\end{figure}

%%%%%%%%%%%%%%%%%%%%%%%%%%%%%%%%%%%%%%%%%%%%%%%%%%%%%%%%%%%
\begin{figure}[h]
\centerline{\epsfxsize=15cm\epsfbox{s2_sqrt_01_01.ps}}
\caption{Simulation \#2. 
D�convolution avec r�gularisation convexe $\sqrt{s^2+x^2}$
avec $s=0.01$ ($\alpha$=0.1, tol�rance = $10^{-6}$).}
\label{fig:s2-sqrt01}
\end{figure}
