%% Simulation No3
%%%%%%%%%%%%%%%%%%%%%%%%%%%%%%%%%%%%%%%%%%%%%%%%%%%%%%%%%%%
\begin{figure}[h]
\centerline{\epsfxsize=7.5cm\epsfbox{s3_yy0.ps}
\epsfxsize=7.5cm\epsfbox{s4_yyb.ps}}
\vskip 1mm
\centerline{\epsfxsize=7.5cm\epsfbox{s4_wiener_1.ps}
\epsfxsize=7.5cm\epsfbox{s4_tikho_1.ps}}
\vskip 1mm
\centerline{\epsfxsize=7.5cm\epsfbox{s4_ggauss_1.ps}
\epsfxsize=7.5cm\epsfbox{s3_mem_01.ps}}
\caption{Simulation \#4. 
De gauche � droite et de haut en bas: 
objet original, 
image filtr�e et bruit� avec SNR=30 dB,
d�convolution par filtre de Wiener ($\alpha$=1, erreur rms=0.0519),
avec r�gularisation de Tikhonov 
($\alpha$=1, it=266, rms=0.0519), 
avec r�gularisation g�n�ralis�e de Gauss
(p=1.1, $\alpha$=1, it=518, rms=0.0519), 
avec maximum d'entropie
($\alpha$=0.1, it=36, rms=0.0519), 
Pour la d�convolution, le crit�re d'arr�t a �t� fix� � $10^{-6}$.}
\label{fig:s4-simu}
\end{figure}

%%%%%%%%%%%%%%%%%%%%%%%%%%%%%%%%%%%%%%%%%%%%%%%%%%%%%%%%%%%
\begin{figure}[h]
\centerline{\epsfxsize=7.5cm\epsfbox{s4_sqrt_1_1.ps}
\epsfxsize=7.5cm\epsfbox{s4_sqrt_10_1.ps}}
\caption{Simulation \#4.
D�convolution avec r�gularisation convexe $\sqrt{s^2+x^2}$
avec $s$=1., $\alpha$=1., it=831, rms=0.0520 (� gauche)
et $s$=1., $\alpha$=10., it=715, rms=0.0520 (� droite)
(tol�rance = $10^{-6}$).}
\end{figure}

%%%%%%%%%%%%%%%%%%%%%%%%%%%%%%%%%%%%%%%%%%%%%%%%%%%%%%%%%%%
\begin{figure}[h]
\centerline{\epsfxsize=7.5cm\epsfbox{s4_gmark_1.ps}}
\caption{Simulation \#4.
D�convolution avec r�gularisation Gauss-Markov 
avec $\alpha$=1., it=536, rms=0.0519 (� gauche)
(tol�rance = $10^{-6}$).}
\end{figure}

