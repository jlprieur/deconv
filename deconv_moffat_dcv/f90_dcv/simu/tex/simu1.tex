%\section{Simulation No2}

%%%%%%%%%%%%%%%%%%%%%%%%%%%%%%%%%%%%%%%%%%%%%%%%%%%%%%%%%%%
\begin{figure}[h]
\centerline{\epsfxsize=9cm\epsfbox{orig.ps}
\hskip-1cm\epsfxsize=9cm\epsfbox{pw_orig.ps}}
\centerline{\epsfxsize=9cm\epsfbox{psf.ps}
\hskip-1cm\epsfxsize=9cm\epsfbox{pw_psf.ps}}
\centerline{\epsfxsize=9cm\epsfbox{filt_orig.ps}
\hskip-1cm\epsfxsize=9cm\epsfbox{pw_filt_orig.ps}}
\centerline{\epsfxsize=9cm\epsfbox{noisy_sig.ps}
\hskip-1cm\epsfxsize=9cm\epsfbox{pw_noisy.ps}}
\caption{Simulation \#1. De haut en bas (� gauche): signal original,
r�ponse impulsionnelle du filtre, signal filtr� et signal filtr� et bruit�.
A droite: spectre de puissance correspondant.}
\label{fig:s1-signal}
\end{figure}

%%%%%%%%%%%%%%%%%%%%%%%%%%%%%%%%%%%%%%%%%%%%%%%%%%%%%%%%%%%
\begin{figure}[h]
\centerline{\epsfxsize=10cm\epsfbox{s1_inverse.ps}
\hskip-1cm\epsfxsize=10cm\epsfbox{s1_inverse_filt.ps}}
\centerline{\epsfxsize=10cm\epsfbox{s1_wiener.ps}
\hskip-1cm\epsfxsize=10cm\epsfbox{s1_wiener_filt.ps}}
\caption{Simulation \#1. 
En haut: simple division spectrale et filtre inverse correspondant.
En bas: deconvolution par filtre de Wiener et filtre correspondant.}
\label{fig:s1-wiener}
\end{figure}

%%%%%%%%%%%%%%%%%%%%%%%%%%%%%%%%%%%%%%%%%%%%%%%%%%%%%%%%%%%
\begin{figure}[h]
\centerline{\epsfxsize=10cm\epsfbox{s1_tikho_0.ps}
\hskip-1cm\epsfxsize=10cm\epsfbox{s1_tikho_05.ps}}
\centerline{\epsfxsize=10cm\epsfbox{s1_tikho_1.ps}
\hskip-1cm\epsfxsize=10cm\epsfbox{s1_tikho_2.ps}}
\caption{Simulation \#1. D�convolution avec r�gularisation de Tikhonov. 
De gauche � droite et de haut en bas: $\alpha$=0 (sans r�gularisation), 
$\alpha$=0.5, $\alpha$=1. et $\alpha$=2 (tol�rance = $10^{-6}$).}
\label{fig:s1-tikho}
\end{figure}

%%%%%%%%%%%%%%%%%%%%%%%%%%%%%%%%%%%%%%%%%%%%%%%%%%%%%%%%%%%
\begin{figure}[h]
\centerline{\epsfxsize=10cm\epsfbox{s1_mem_05.ps}
\hskip-1cm\epsfxsize=10cm\epsfbox{s1_mem_1.ps}}
\centerline{\epsfxsize=10cm\epsfbox{s1_mem_2.ps}
\hskip-1cm\epsfxsize=10cm\epsfbox{s1_mem_5.ps}}
\caption{Simulation \#1. D�convolution avec la m�thode du Maximum d'Entropie. 
De gauche � droite et de haut en bas: $\alpha$=0.5, 
$\alpha$=1., $\alpha$=2. et $\alpha$=5 (tol�rance = $10^{-6}$).}
\label{fig:s1-mem}
\end{figure}

%%%%%%%%%%%%%%%%%%%%%%%%%%%%%%%%%%%%%%%%%%%%%%%%%%%%%%%%%%%
\begin{figure}[h]
\centerline{\epsfxsize=10cm\epsfbox{s1_gmark_5.ps}
\hskip-1cm\epsfxsize=10cm\epsfbox{s1_gmark_10.ps}}
\caption{Simulation \#1. D�convolution avec r�gularisation de Gauss-Markov: 
A gauche: $\alpha$=5, � droite: $\alpha$=10 (tol�rance = $10^{-6}$).}
\label{fig:s1-gmark}
\end{figure}

%%%%%%%%%%%%%%%%%%%%%%%%%%%%%%%%%%%%%%%%%%%%%%%%%%%%%%%%%%%
\begin{figure}[h]
\centerline{\epsfxsize=10cm\epsfbox{s1_sqrt_10_001.ps}
\hskip-1cm\epsfxsize=10cm\epsfbox{s1_sqrt_10_05.ps}}
\caption{Simulation \#1. 
D�convolution avec r�gularisation convexe en $\sqrt{s^2+x^2}$
avec $s$=0.001 (� gauche) et $s$=0.05 (� droite). Dans les deux cas: 
$\alpha$=10. et tol�rance = $10^{-6}$.}
\label{fig:s1-sqrt}
\end{figure}

