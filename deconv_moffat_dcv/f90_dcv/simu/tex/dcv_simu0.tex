%\documentclass{article}
\documentclass[12pt]{book}

%%% BOF  FONTS
% In /usr/share/texmf/tex/latex/seminar, fonts are helvetica from semhelv.sty :
% But there are other possibilities (in /usr/share/texmf/tex/latex/psnfss):
% (should check that \rmdefault is redefined in those *.sty files...)
%
% For Times fonts
\usepackage{times}
% Bookman: nice for titles, but still similar to Times and not easy to read
%\usepackage{bookman}
% Chancery: original, look a bit like gothic, but too light.
%\usepackage{chancery}
% Charter: similar to Times, but more square; a possible alternative to it.
%\usepackage{charter}
% NewCent: better than Times (like Times, but heavier, hence easier to read).
%\usepackage{newcent}
% Palatino: similar to Times, but not easy to read
%\usepackage{palatino}
% Utopia: similar to Times, but not easy to read
%\usepackage{utopia}
%%% EOF  FONTS


\voffset=-15mm
\hoffset=-5mm
\evensidemargin=10mm
\textwidth=156mm
\textheight=250mm

\usepackage{epsf}
% To include eepic files:
\usepackage{epic,eepic}

% To display math symbols, R, N, Z, etc
\usepackage{xspace}
\usepackage{bbold}

%%% Pour les accents:
\usepackage[latin1]{inputenc}
\usepackage[T1]{fontenc}
\usepackage[french]{babel}

% For eps graphics
\usepackage{psfig}
% I add the directory ./figs for figures:
\psfigurepath{figs}

% Pour les figures compl\`etes:
\psfull
% Pour le cadre seulement des figures:
%\psdraft

%\includeonly{simu5}

\begin{document}

\chapter{Mise en \oe uvre de m�thodes de d�convolution}

\centerline{\large \sl Jean-Louis Prieur, LA2T, OMP, UPS-CNRS} 

\centerline{\large \sl Version \today}

\bigskip
Dans ce chapitre, je d�cris quelques m�thodes de d�convolution
que j'ai mises en \oe uvre de fa�on concr�te.
L'id�e initiale �tait d'�crire un logiciel 
permettant de tester et comparer plusieurs 
m�thodes de d�convolution, parmi les plus connues en analyse
num�rique, afin de d�convoluer des images astronomiques. 

J'ai commenc� par reproduire les simulations � une dimension
d'Herv� Carfantan (simulations \#~1 et~2), 
initialement faites en "matlab", puis
je les ai �tendues � 2 dimensions.
Ces programmes ont d'abord �t� �crits en Fortran 95, car ce langage 
est particuli�rement bien adapt� au calcul matriciel. Cependant, 
des probl�mes de compatibilit� entre diff�rents compilateurs, m'ont
contraint � les traduire en C.

Enfin, dans la section~\ref{sec:dcv-diane}, je d�cris avec plus de d�tails
la m�thode r�gularis�e avec contrainte de support d�velopp�e par notre �quipe.

%%%%%%%%%%
\section{Simulation des donn�es}
\label{sec:dcv-simu}

Pour la simulation \#~1, l'objet $y_0(i)$ 
et la r�ponse impulsionnelle $h(i)$, qui est un filtre d'int�gration,
sont directement g�n�r�s par le programme {\tt "dcv\_test1.f90" option=1}
(cf. Fig.~\ref{fig:s1-signal}).

Pour la simulation \#~2, l'objet $y_0(i)$ est un spectre de raies 
et la r�ponse impulsionnelle $h(i)$ est une fonction ondelette
(cf. Fig.~\ref{fig:s2-signal}).
Il sont g�n�r�s par le programme {\tt "dcv\_test1.f90" option=2} � partir
respectivement des fichiers {\tt bgg.asc} et {\tt ricker\_ri.asc}. 

Dans le cas � 2 dimensions (simulation \#~5),
l'objet $y_0(i,j)$ et la r�ponse impulsionnelle $h(i,j)$ sont g�n�r�s 
par le programme {\tt "object\_cerga"}
d�riv� d'un programme de P.~Cruzal�bes. 
%% Simulation 3 et 4: �chec, car il existe un fond non uniforme
% L'objet est constitu� d'un disque assombri sur les bords, sur lequel sont
% superpos�s des d�tails � haute
% r�solution spatiale (Fig.~\ref{fig:s3-simu}a). 
L'objet est constitu� d'un anneau et d'autres d�tails � haute
r�solution spatiale (Fig.~\ref{fig:s5-simu}a). 
La r�ponse
impulsionnelle correspond au premier lobe d'une fonction d'Airy.

L'objet est tout d'abord convolu� par la r�ponse impulsionnelle:
$y = h \star y_0$, puis bruit�. 

Pour g�n�rer le signal bruit� $y_b$ (Fig.~\ref{fig:s5-simu}b), 
pour chaque pixel $(i,j)$ 
on ajoute � la valeur du signal $y(i,j)$ 
un nombre al�atoire $r_{i,j}$ 
(uniforme entre 0 et 1, et donc de variance 1/12) 
centr� et multipli� par un facteur de normalisation not� $g$:
\begin{equation}
y_b(i,j) = h \star y_0(i,j) + g \times (r_{i,j} - 0.5)
\qquad {\rm avec} \quad g^2 = 12 \times 10^{- {\it SNR}/10} 
\times {\sum_{i,j} y_0(i,j)^2 \over N} 
\end{equation}
o� $N$ est le nombre total de pixels 
et {\it SNR} est la valeur du rapport signal sur bruit en dB (d�fini
ici comme le rapport de la variance du signal sur celle du bruit).

%%%%%%%%%%%%%%%%%%%%%%%%%%%%%%%%%%%%%%%%%%%%%%%%%%%%%%%%%%%%%%
\section{M�thodes disponibles avec dcv\_deconv}

On minimise la fonctionnelle suivante:
\begin{equation}
q(x) = || y_b - H x ||^2 + \alpha \, \Phi(x) 
\end{equation}
o� $H$ est l'op�rateur de convolution 
et $\Phi(x)$ est une fonction de r�gularisation.
soit encore, en notant $Y$ et $X$ les matrices colonnes correspondant
respectivement aux fonctions $y_b$ et $x$:
$$ 
q(X) = ( Y - H X)^T ( Y - H X) + \alpha \, \Phi(X) 
$$

Le gradient de $q(x)$ est alors:
\begin{equation}
{\rm d}q(X) = - 2 H^T ( Y - H X) + \alpha \, {\rm d}\Phi(X) 
\end{equation}

Notons que $H x = U^{-1} \hat{h} \times U x$ 
et $H^T z = U^{-1} \hat{h'} \times U z$, avec $h'(x)=h(-x)$ et
en notant $U$ l'op�rateur de transform�e de Fourier. 

Mes deux programmes {\tt "dcv\_deconv\_1D"},
{\tt "dcv\_deconv\_2D"} permettent de d�convoluer 
par un filtrage de Wiener ou simple division spectrale, ou bien 
d'utiliser les fonctions de r�gularisation d�crites 
dans le tableau~\ref{tab:phi}.

%%%%%%%%%%%%%%%%%%%%%%%%%%%%%%%%%%%%%%%%%%%%%%%%%%%%%
\begin{table}[h]
\centering
{\small
\begin{tabular}{|c|c|c|}
\hline
 & & \cr
M�thode & $\Phi(x)$ & d$\Phi(x)$ \cr
 & & \cr
\hline
 & & \cr
%%%
Tikhonov & $\displaystyle \sum_{i,j} x_{i,j}^2$ &  $2 \, x_{i,j}$ \cr 
 & & \cr
\hline
 & & \cr
%%%
Gauss g�n�ralis� & $\displaystyle \sum_{i,j} |x_{i,j}|^{1.1}$ 
&  $0.1 \, x_{i,j}^{0.1} \times sign(x_{i,j})$ \cr 
 & & \cr
\hline
 & & \cr
%%%
Max. d'entropie 
& $\displaystyle \sum_{i,j} x_{i,j} \log\left[x_{i,j}\right]$ 
&  $1  + \log\left[x_{i,j}\right]$ \cr 
 & & \cr
\hline
 & & \cr
%%%
$\sqrt{s^2+x^2}$ & $\displaystyle \sum_{i,j} \sqrt{s^2+x_{i,j}^2}$ 
&  $\displaystyle {x_{i,j} \over \sqrt{s^2+x_{i,j}^2}}$ \cr 
 & & \cr
\hline
 & & \cr
%%%
Gauss-Markov & $\displaystyle \sum_{i,j} 
\left[ x_{i+1,j} - x_{i,j} \right]^2  
+ \left[ x_{i,j+1} - x_{i,j} \right]^2$ 
& $-2 \, \left[ x_{i+1,j} - x_{i,j} \right]  
+ 2 \, \left[ x_{i,j} - x_{i-1,j} \right] + \ldots$ \cr 
 & & \cr
%%%
\hline
\end{tabular}
% End small fonts
}
\caption{Fonctions de r�gularisation et gradients associ�s, pour
les m�thodes actuellement disponibles.}
\label{tab:phi}
\end{table}

La minimisation peut �tre effectu�e par deux m�thodes:
gradients conjugu�s ou L-BFGS-B.

La m�thode des gradients
conjugu�s utilis�e est celle de Polak-Ribi�re 
(d�riv�e de la m�thode de Fletcher-Reeves) 
avec la fonction {\sl ``frprmn"} de {\sl ``Numerical Recipees"}. 
J'ai modifi� le crit�re d'arr�t
et je fais un double test sur la variation relative de $q(x)$ 
et de la norme L2 de $x$. 
La minimisation dans une direction donn�e se fait {\sl "dbrent"}
qui utilise une m�thode d�riv�e de celle 
de Brent avec une interpolation parabolique.

La m�thode L-BFGS-B 
(R.H. Byrd, P. Lu, J. Nocerdal, C. Zhu, ``A limited
memory algorithm for bound constrained optimization'',
SIAM Journal on Scientific Computing, 16, 5, pp 1190-1208) 
est une m�thode d'optimisation avec contraintes
de type ``quasi-Newton'', qui fait une utilisation optimale de
la m�moire et qui est bien adapt�e � la r�solution de grands syst�mes.
Elle utilise aussi le gradient de la fonctionnelle $f$ � minimiser, mais
la connaissance du Hessien n'est pas n�cessaire. A chaque it�ration,
elle calcule une approximation du Hessien qui permet de d�finir un mod�le
quadratique de $f$. Une direction de recherche est d�termin�e en deux �tapes:
d'abord, une m�thode de projection du gradient permet d'identifier
les variables ``actives'' (i.e., celles qui seront maintenues
constantes sur leur limite), et ensuite le mod�le quadratique est 
minimis� par rapport aux variables libres. La direction de recherche
est alors d�finie par le vecteur ayant pour extr�mit�s le point
de l'it�ration pr�c�dente et le point de minimisation approximatif 
ainsi obtenu. Enfin une minimisation compl�te est faite le long
de cette direction de recherche.

%%%%%%%%%%%%%%%%%%%%%%%%%%%%%%%%%%%%%%%%%%%%%%%%%%%%%%%%%%%%%%
\section{R�sultats du traitement avec dcv\_deconv}

Nous pr�sentons maintenant les r�sultats obtenus lors
de la mise en \oe uvre du programme {\tt dcv\_deconv} dans les trois
cas d�crits dans la section~\ref{sec:dcv-simu}:
\begin{itemize}
\item[Simulation \#1:]{Cas monodimensionnel, avec un profil discontinu; 
}
\item[Simulation \#2:]{Cas monodimensionnel, avec un spectre de raies;
}
\item[Simulation \#5:]{Cas bidimensionnel, avec un anneau et des sources 
ponctuelles.
}
\end{itemize}


%% Simulations #1: cas monodimensionnel (profil un peu complexe) 
%\section{Simulation No2}

%%%%%%%%%%%%%%%%%%%%%%%%%%%%%%%%%%%%%%%%%%%%%%%%%%%%%%%%%%%
\begin{figure}[h]
\centerline{\epsfxsize=9cm\epsfbox{orig.ps}
\hskip-1cm\epsfxsize=9cm\epsfbox{pw_orig.ps}}
\centerline{\epsfxsize=9cm\epsfbox{psf.ps}
\hskip-1cm\epsfxsize=9cm\epsfbox{pw_psf.ps}}
\centerline{\epsfxsize=9cm\epsfbox{filt_orig.ps}
\hskip-1cm\epsfxsize=9cm\epsfbox{pw_filt_orig.ps}}
\centerline{\epsfxsize=9cm\epsfbox{noisy_sig.ps}
\hskip-1cm\epsfxsize=9cm\epsfbox{pw_noisy.ps}}
\caption{Simulation \#1. De haut en bas (� gauche): signal original,
r�ponse impulsionnelle du filtre, signal filtr� et signal filtr� et bruit�.
A droite: spectre de puissance correspondant.}
\label{fig:s1-signal}
\end{figure}

%%%%%%%%%%%%%%%%%%%%%%%%%%%%%%%%%%%%%%%%%%%%%%%%%%%%%%%%%%%
\begin{figure}[h]
\centerline{\epsfxsize=10cm\epsfbox{s1_inverse.ps}
\hskip-1cm\epsfxsize=10cm\epsfbox{s1_inverse_filt.ps}}
\centerline{\epsfxsize=10cm\epsfbox{s1_wiener.ps}
\hskip-1cm\epsfxsize=10cm\epsfbox{s1_wiener_filt.ps}}
\caption{Simulation \#1. 
En haut: simple division spectrale et filtre inverse correspondant.
En bas: deconvolution par filtre de Wiener et filtre correspondant.}
\label{fig:s1-wiener}
\end{figure}

%%%%%%%%%%%%%%%%%%%%%%%%%%%%%%%%%%%%%%%%%%%%%%%%%%%%%%%%%%%
\begin{figure}[h]
\centerline{\epsfxsize=10cm\epsfbox{s1_tikho_0.ps}
\hskip-1cm\epsfxsize=10cm\epsfbox{s1_tikho_05.ps}}
\centerline{\epsfxsize=10cm\epsfbox{s1_tikho_1.ps}
\hskip-1cm\epsfxsize=10cm\epsfbox{s1_tikho_2.ps}}
\caption{Simulation \#1. D�convolution avec r�gularisation de Tikhonov. 
De gauche � droite et de haut en bas: $\alpha$=0 (sans r�gularisation), 
$\alpha$=0.5, $\alpha$=1. et $\alpha$=2 (tol�rance = $10^{-6}$).}
\label{fig:s1-tikho}
\end{figure}

%%%%%%%%%%%%%%%%%%%%%%%%%%%%%%%%%%%%%%%%%%%%%%%%%%%%%%%%%%%
\begin{figure}[h]
\centerline{\epsfxsize=10cm\epsfbox{s1_mem_05.ps}
\hskip-1cm\epsfxsize=10cm\epsfbox{s1_mem_1.ps}}
\centerline{\epsfxsize=10cm\epsfbox{s1_mem_2.ps}
\hskip-1cm\epsfxsize=10cm\epsfbox{s1_mem_5.ps}}
\caption{Simulation \#1. D�convolution avec la m�thode du Maximum d'Entropie. 
De gauche � droite et de haut en bas: $\alpha$=0.5, 
$\alpha$=1., $\alpha$=2. et $\alpha$=5 (tol�rance = $10^{-6}$).}
\label{fig:s1-mem}
\end{figure}

%%%%%%%%%%%%%%%%%%%%%%%%%%%%%%%%%%%%%%%%%%%%%%%%%%%%%%%%%%%
\begin{figure}[h]
\centerline{\epsfxsize=10cm\epsfbox{s1_gmark_5.ps}
\hskip-1cm\epsfxsize=10cm\epsfbox{s1_gmark_10.ps}}
\caption{Simulation \#1. D�convolution avec r�gularisation de Gauss-Markov: 
A gauche: $\alpha$=5, � droite: $\alpha$=10 (tol�rance = $10^{-6}$).}
\label{fig:s1-gmark}
\end{figure}

%%%%%%%%%%%%%%%%%%%%%%%%%%%%%%%%%%%%%%%%%%%%%%%%%%%%%%%%%%%
\begin{figure}[h]
\centerline{\epsfxsize=10cm\epsfbox{s1_sqrt_10_001.ps}
\hskip-1cm\epsfxsize=10cm\epsfbox{s1_sqrt_10_05.ps}}
\caption{Simulation \#1. 
D�convolution avec r�gularisation convexe en $\sqrt{s^2+x^2}$
avec $s$=0.001 (� gauche) et $s$=0.05 (� droite). Dans les deux cas: 
$\alpha$=10. et tol�rance = $10^{-6}$.}
\label{fig:s1-sqrt}
\end{figure}


%% Simulations #2: cas monodimensionnel (spectre de raies) 
%\section{Simulation No1}

%%%%%%%%%%%%%%%%%%%%%%%%%%%%%%%%%%%%%%%%%%%%%%%%%%%%%%%%%%%
\begin{figure}[h]
\centerline{\epsfxsize=9cm\epsfbox{s2_orig.ps}
\hskip-1cm\epsfxsize=9cm\epsfbox{s2_pw_orig.ps}}
\centerline{\epsfxsize=9cm\epsfbox{s2_psf.ps}
\hskip-1cm\epsfxsize=9cm\epsfbox{s2_pw_psf.ps}}
\centerline{\epsfxsize=9cm\epsfbox{s2_filt_orig.ps}
\hskip-1cm\epsfxsize=9cm\epsfbox{s2_pw_filt_orig.ps}}
\centerline{\epsfxsize=9cm\epsfbox{s2_noisy_sig.ps}
\hskip-1cm\epsfxsize=9cm\epsfbox{s2_pw_noisy.ps}}
\caption{Simulation \#2. De bas en haut (� gauche): signal original,
r�ponse impulsionnelle du filtre, signal filtr� et signal filtr� et bruit�.
A droite: spectre de puissance correspondant.}
\label{fig:s2-signal}
\end{figure}

%%%%%%%%%%%%%%%%%%%%%%%%%%%%%%%%%%%%%%%%%%%%%%%%%%%%%%%%%%%
\begin{figure}[h]
\centerline{\epsfxsize=10cm\epsfbox{s2_inverse.ps}
\hskip-1cm\epsfxsize=10cm\epsfbox{s2_inverse_filt.ps}}
\centerline{\epsfxsize=10cm\epsfbox{s2_wiener.ps}
\hskip-1cm\epsfxsize=10cm\epsfbox{s2_wiener_filt.ps}}
\caption{Simulation \#2. 
En haut: simple division spectrale et filtre inverse correspondant.
En bas: deconvolution par filtre de Wiener et filtre correspondant.}
\label{fig:s2-wiener}
\end{figure}

%%%%%%%%%%%%%%%%%%%%%%%%%%%%%%%%%%%%%%%%%%%%%%%%%%%%%%%%%%%
\begin{figure}[h]
\centerline{\epsfxsize=15cm\epsfbox{s2_tikho_01.ps}}
\caption{Simulation \#2. 
D�convolution avec r�gularisation de Tikhonov 
avec $\alpha$=0.1 (tol�rance = $10^{-6}$).}
\label{fig:s2-tikhonov}
\end{figure}

%%%%%%%%%%%%%%%%%%%%%%%%%%%%%%%%%%%%%%%%%%%%%%%%%%%%%%%%%%%
\begin{figure}[h]
\centerline{\epsfxsize=15cm\epsfbox{s2_ggauss_01.ps}}
\caption{Simulation \#2. 
D�convolution avec r�gularisation de Gauss g�n�ralis� 
avec p=1.1 avec $\alpha$=0.1 (tol�rance = $10^{-6}$).}
\label{fig:s2-ggauss}
\end{figure}

%%%%%%%%%%%%%%%%%%%%%%%%%%%%%%%%%%%%%%%%%%%%%%%%%%%%%%%%%%%
\begin{figure}[h]
\centerline{\epsfxsize=15cm\epsfbox{s2_sqrt_01_001.ps}}
\caption{Simulation \#2. 
D�convolution avec r�gularisation convexe $\sqrt{s^2+x^2}$
avec $s=0.001$ ($\alpha$=0.1, tol�rance = $10^{-6}$).}
\label{fig:s2-sqrt001}
\end{figure}

%%%%%%%%%%%%%%%%%%%%%%%%%%%%%%%%%%%%%%%%%%%%%%%%%%%%%%%%%%%
\begin{figure}[h]
\centerline{\epsfxsize=15cm\epsfbox{s2_sqrt_01_01.ps}}
\caption{Simulation \#2. 
D�convolution avec r�gularisation convexe $\sqrt{s^2+x^2}$
avec $s=0.01$ ($\alpha$=0.1, tol�rance = $10^{-6}$).}
\label{fig:s2-sqrt01}
\end{figure}

%% Simulations #3: avec fond non uniforme: �chec!
%%% Simulation No3
%%%%%%%%%%%%%%%%%%%%%%%%%%%%%%%%%%%%%%%%%%%%%%%%%%%%%%%%%%%
\begin{figure}[h]
\centerline{\epsfxsize=7.5cm\epsfbox{s3_yy0.ps}
\epsfxsize=7.5cm\epsfbox{s3_yyb.ps}}
\vskip 1mm
\centerline{\epsfxsize=7.5cm\epsfbox{s3_wiener_08.ps}
\epsfxsize=7.5cm\epsfbox{s3_tikho_1.ps}}
\vskip 1mm
\centerline{\epsfxsize=7.5cm\epsfbox{s3_ggauss_1.ps}
\epsfxsize=7.5cm\epsfbox{s3_mem_01.ps}}
\caption{Simulation \#3. 
De gauche � droite et de haut en bas: 
objet original, 
image filtr�e et bruit� avec SNR=20 dB,
d�convolution par filtre de Wiener ($\alpha$=0.8, erreur rms=0.0521),
avec r�gularisation de Tikhonov 
($\alpha$=1, it=317, rms=0.0520), 
avec r�gularisation g�n�ralis�e de Gauss
(p=1.1, $\alpha$=1, it=534, rms=0.0521), 
avec maximum d'entropie
($\alpha$=0.1, it=46, rms=0.0519), 
Pour la d�convolution, le crit�re d'arr�t a �t� fix� � $10^{-6}$.}
\label{fig:s3-simu}
\end{figure}

%%%%%%%%%%%%%%%%%%%%%%%%%%%%%%%%%%%%%%%%%%%%%%%%%%%%%%%%%%%
\begin{figure}[h]
\centerline{\epsfxsize=7.5cm\epsfbox{s3_sqrt_1_1.ps}
\epsfxsize=7.5cm\epsfbox{s3_sqrt_10_1.ps}}
\caption{Simulation \#3.
D�convolution avec r�gularisation convexe $\sqrt{s^2+x^2}$
avec $s$=1., $\alpha$=1., it=675, rms=0.0522 (� gauche)
et $s$=1., $\alpha$=10., it=369, rms=0.0520 (� droite)
(tol�rance = $10^{-6}$).}
\end{figure}

%%%%%%%%%%%%%%%%%%%%%%%%%%%%%%%%%%%%%%%%%%%%%%%%%%%%%%%%%%%
\begin{figure}[h]
\centerline{\epsfxsize=7.5cm\epsfbox{s3_gmark_1.ps}}
%\epsfxsize=7.5cm\epsfbox{s3_sqrt_1_01.ps}}
\caption{Simulation \#3.
D�convolution avec r�gularisation Gauss-Markov 
avec $\alpha$=1., it=354, rms=0.0521 (� gauche)
%et $s$=0.01, $\alpha$=1., it=74, rms=0.0530 (� droite)
(tol�rance = $10^{-6}$).}
\end{figure}


%% Simulations #4: avec fond non uniforme: �chec!
%%% Simulation No3
%%%%%%%%%%%%%%%%%%%%%%%%%%%%%%%%%%%%%%%%%%%%%%%%%%%%%%%%%%%
\begin{figure}[h]
\centerline{\epsfxsize=7.5cm\epsfbox{s3_yy0.ps}
\epsfxsize=7.5cm\epsfbox{s4_yyb.ps}}
\vskip 1mm
\centerline{\epsfxsize=7.5cm\epsfbox{s4_wiener_1.ps}
\epsfxsize=7.5cm\epsfbox{s4_tikho_1.ps}}
\vskip 1mm
\centerline{\epsfxsize=7.5cm\epsfbox{s4_ggauss_1.ps}
\epsfxsize=7.5cm\epsfbox{s3_mem_01.ps}}
\caption{Simulation \#4. 
De gauche � droite et de haut en bas: 
objet original, 
image filtr�e et bruit� avec SNR=30 dB,
d�convolution par filtre de Wiener ($\alpha$=1, erreur rms=0.0519),
avec r�gularisation de Tikhonov 
($\alpha$=1, it=266, rms=0.0519), 
avec r�gularisation g�n�ralis�e de Gauss
(p=1.1, $\alpha$=1, it=518, rms=0.0519), 
avec maximum d'entropie
($\alpha$=0.1, it=36, rms=0.0519), 
Pour la d�convolution, le crit�re d'arr�t a �t� fix� � $10^{-6}$.}
\label{fig:s4-simu}
\end{figure}

%%%%%%%%%%%%%%%%%%%%%%%%%%%%%%%%%%%%%%%%%%%%%%%%%%%%%%%%%%%
\begin{figure}[h]
\centerline{\epsfxsize=7.5cm\epsfbox{s4_sqrt_1_1.ps}
\epsfxsize=7.5cm\epsfbox{s4_sqrt_10_1.ps}}
\caption{Simulation \#4.
D�convolution avec r�gularisation convexe $\sqrt{s^2+x^2}$
avec $s$=1., $\alpha$=1., it=831, rms=0.0520 (� gauche)
et $s$=1., $\alpha$=10., it=715, rms=0.0520 (� droite)
(tol�rance = $10^{-6}$).}
\end{figure}

%%%%%%%%%%%%%%%%%%%%%%%%%%%%%%%%%%%%%%%%%%%%%%%%%%%%%%%%%%%
\begin{figure}[h]
\centerline{\epsfxsize=7.5cm\epsfbox{s4_gmark_1.ps}}
\caption{Simulation \#4.
D�convolution avec r�gularisation Gauss-Markov 
avec $\alpha$=1., it=536, rms=0.0519 (� gauche)
(tol�rance = $10^{-6}$).}
\end{figure}


%% Simulations #5: avec fond uniforme et null: OK! 
%% Simulation No3
%%%%%%%%%%%%%%%%%%%%%%%%%%%%%%%%%%%%%%%%%%%%%%%%%%%%%%%%%%%
\begin{figure}[h]
\centerline{\epsfxsize=7.5cm\epsfbox{s5_yy0.ps}
\epsfxsize=7.5cm\epsfbox{s5_yyb.ps}}
\vskip 1mm
\centerline{\epsfxsize=7.5cm\epsfbox{s5_wiener_1.ps}
\epsfxsize=7.5cm\epsfbox{s5_tikho_10.ps}}
\vskip 1mm
\centerline{\epsfxsize=7.5cm\epsfbox{s5_ggauss_10.ps}
\epsfxsize=7.5cm\epsfbox{s5_mem_1e-5.ps}}
\caption{Simulation \#5. 
De gauche � droite et de haut en bas: 
objet original, 
image filtr�e et bruit� avec SNR=13 dB,
d�convolution par filtre de Wiener ($\alpha$=1, erreur rms=0.0100),
avec r�gularisation de Tikhonov ($\alpha$=10., it=144, rms=0.0097), 
avec r�gul. g�n�ralis�e de Gauss (p=1.1, $\alpha$=10, it=1000, rms=0.0103), 
avec maximum d'entropie ($\alpha$=1e-5, it=59, rms=0.0099), 
Pour la d�convolution, le crit�re d'arr�t a �t� fix� � $10^{-6}$.}
\label{fig:s5-simu}
\end{figure}

%%%%%%%%%%%%%%%%%%%%%%%%%%%%%%%%%%%%%%%%%%%%%%%%%%%%%%%%%%%
\begin{figure}[h]
\centerline{\epsfxsize=7.5cm\epsfbox{s5_sqrt_10_01.ps}
\epsfxsize=7.5cm\epsfbox{s5_sqrt_10_1.ps}}
\vskip 1mm
\centerline{\epsfxsize=7.5cm\epsfbox{s5_sqrt_10_01_lin.ps}
\epsfxsize=7.5cm\epsfbox{s5_sqrt_10_1_lin.ps}}
\caption{Simulation \#5.
D�convolution avec r�gularisation convexe $\sqrt{s^2+x^2}$
avec $s$=0.1, $\alpha$=10., it=1000., rms=0.0104(� gauche)
et $s$=1., $\alpha$=10., it=792, rms=0.0101 (� droite)
(tol�rance = $10^{-6}$). En haut, �chelle logarithmique,
en bas, �chelle lin�aire.}
\end{figure}

%%%%%%%%%%%%%%%%%%%%%%%%%%%%%%%%%%%%%%%%%%%%%%%%%%%%%%%%%%%
\def\comments{
\begin{figure}[h]
\centerline{\epsfxsize=7.5cm\epsfbox{s4_gmark_1.ps}}
\caption{Simulation \#5.
D�convolution avec r�gularisation Gauss-Markov 
avec $\alpha$=1., it=536, rms=0.0519 (� gauche)
(tol�rance = $10^{-6}$).}
\end{figure}
}

%%%%%%%%%%%%%%%%%%%%%%%%%%%%%%%%%%%%%%%%%%%%%%%%%%%%%%%%%%%%
\def\Rr{R}
\def\Cc{C}
\def\Annexmath{A}
\def\Annexmcarres{B}
%%%%%%%%%%%%%%%%%%%%%%%%%%%%%%%%%%%%%%%%%%%%%%%%%%%%%%%%%%%%%%
\section{D�convolution r�gularis�e DIANE/WIPE}
\label{sec:dcv-diane}

Dans ce chapitre, nous pr�senterons 
la m�thode de d�convolution r�gularis�e propos�e par 
Lannes {\it et al},~1987, et Roques (1987),
que j'ai adapt�e aux probl�mes que j'ai rencontr�s en astronomie
(interf�rom�trie des tavelures, images d'optique adaptative, etc).

%%%%%%%%%%%%%%%%%%%%%%%%%%%%%%%%%%%%%%%%%%%%%%%%%%%%%%%%%%%
\subsection{D�convolution en pr�sence de bruit}

Consid�rons l'image $\psi_i(x)$ d'un objet $\phi_0(x)$ par
un syst�me imageur lin�aire de r�ponse impulsionnelle
$h(x)$. En pr�sence de bruit, nous avons une relation 
du type: 
  $$  %%%% 
\psi_i (x) = \phi_0 \star h\,(x) + b(x)
\eqno({\rm II.33})
  $$
o� $b(x)$ repr�sente le terme de bruit (erreurs de mesure, bruit 
de d�tection, mauvaise estimation de $h(x)$, etc).
La d�convolution consiste � inverser cette �quation 
et � restituer la version la plus fid�le possible 
de l'objet initial $\phi_0$.

Par transform�e de Fourier:
  $$  %%%% 
\hat{\psi}_i (u) = \hat{\phi}_0(u) \; \hat{h}(u) + \hat{b}(u)
\eqno({\rm II.34})
  $$

La restauration du spectre de l'objet $\hat{\phi}(u)$ est donc
probl�matique pour les fr�quences $u$ pour lesquelles 
$\hat{b} (u) / \hat{h}(u)$ est tr�s grand devant 
$\hat{\phi}_0 (u)$. Or tous les syst�mes physiques pr�sentent 
une fr�quence de coupure $u_c$ au-del� de laquelle la r�ponse est 
nulle. La mesure du spectre de l'objet est donc impossible
au-del� de $u_c$. De mani�re g�n�rale, on devra limiter la 
r�solution pour augmenter la fiabilit� de l'objet reconstruit
$\phi_r(x)$. Il existe donc un compromis entre fiabilit� et r�solution.
Ce concept est trait� de mani�re explicite dans la d�convolution r�gularis�e 
propos�e par Lannes {\it et al.},~(1987), que nous allons bri�vement 
d�crire dans les paragraphes suivants.
              
%%%%%%%%%%%%%%%%%%%%%%%%%%%%%%%%%%%%%%%%%%%%%%%%%%%%%%%%%%%
\subsection{Espaces en jeu et op�rateur d'imagerie $A$}

Les observations conduisent � une image brute not�e
$\psi'_{i}$ appartenant � {\sl l'espace image} $F'_{i}$
(cf.~Fig.~\ref{fig:decdiag}).
Le but de la d�convolution est d'obtenir le maximum 
d'information sur la fonction 
objet $\phi_o$ appartenant � {\sl l'espace objet} not� $E_o$.

Nous nous limiterons au cas o� l'objet est d�fini
sur un support born� de $\Rr^{p}$ 
(dans le cas d'imagerie bidimensionnelle, $p\!=\!2$,
en spectroscopie, $p\!=\!1$, cf.~\S IV.4). 
L'espace objet $E_o$ est donc 
$L^2(V_0)$, i.e., l'espace de Hilbert r�el (cf.~Annexe~\Annexmath)
des fonctions de carr� sommable
� valeurs dans $\Rr$ dont le support est $V_0$, 
avec $ V_o \subset \Rr^p$.

Lors de la r�duction des donn�es, pour att�nuer 
les imperfections du syst�me 
d'acquisition (par exemple correction de distorsion g�om�trique ou
de d�fauts cosm�tiques), cette image $\psi'_{i}$ est 
transform�e en une autre image $\psi_i$, par un op�rateur $T$ 
(g�n�ralement irr�versible). L'espace image de $F'_i$ par $T$
sera appel� {\sl espace image d'entr�e} et not� $F_i$.

On se limitera ici au cas o� il existe une 
relation objet-image de la forme:
  $$  %%%% 
\psi_s  = A \phi_s
\eqno({\rm II.35})
  $$
o� $A$ est une application lin�aire continue 
de $E$ dans $F_i$, que l'on appellera {\sl op�rateur d'imagerie} 
et on posera $F \! \equiv \! A E$. On cherchera donc la fonction 
$\psi_r$ de $F$ qui soit le plus proche de~$\psi_i$, l'op�rateur 
correspondant sera not� $R$: 
  $$  %%%% 
\psi_r  = R \psi_i
\eqno({\rm II.36})
  $$
Si le crit�re de proximit� est la norme de $F_i$, alors $R$
est l'op�rateur de {\sl projection orthogonale de $F_i$ sur $F$}.

Lorsque $A$ est bijectif de $E$ sur $F$, 
l'objet reconstruit $\phi_r$ est d�fini par la relation:
$\phi_r \! = \! A^{-1} \psi_r$.
Si $A^{-1}$ n'est pas continu, le probl�me est 
{\sl mal conditionn�}, puisque une erreur de reconstruction image
$(\psi_r - \psi_s)$ born�e ne se traduira pas forc�ment au terme 
de la restauration par une erreur de reconstruction objet 
$(\phi_r - \phi_s)$ �galement born�e.

%%%%%%%%%%%%%%%%%%%%%%%%%%%%%%%%%%%%%%%%%%%%%%%%%%%%%%%%%%%
\subsection{N�cessit� d'une limitation en r�solution} 

Soit $U$ l'op�rateur de transform�e de Fourier
sur $L_{\Cc}^2(\Rr^p)$, l'espace de Hilbert des fonctions complexes d�finies
sur $\Rr^p$ $L_{\Cc}^2(\Rr^p)$ et soit $U^*$ l'op�rateur adjoint.
En se r�f�rant aux ensembles et op�rateurs
d�finis dans le paragraphe pr�c�dent
(\S II.3.b et Fig~\ref{fig:decdiag}), 
la {\sl d�convolution avec extrapolation compl�te},
i.e., sans limitation de r�solution, est caract�ris�e par les 
conditions suivantes:

\smallskip
\vbox{\parindent=0pt \vrule \hskip 1truemm \vrule \hskip1em 
\vbox{%
$E_0 \! = \! E \! = \! L^2(V_0)$,
\hfil\break
$ S \! = \! Id_E \qquad(\Rightarrow \phi_s \! = \! \phi_0 $ et $ V$  
support de $\phi_s$ est �gal � $ V_0$)
\hfil\break
$F_{i}^{'} \! = \! F_i \! = \! U^*L^2(H)$  o� $H \subset \Rr^p$
\hfil\break
et $U^* L^2(H)$ est l'espace de Hilbert 
r�el des fonctions de carr� sommable 
\hfil\break
� valeurs dans \Cc  dont la transform�e de Fourier est � support sur $H$ 
\hfil\break
$ T \! = \! Id_{Fi} \;\;(\Rightarrow \psi_{i}^{'} \! = \!  \psi_i)$   
$ A: E \mapsto F$ d�fini par $ A \phi \! = \! U^* h U 
\phi $}}

\bigskip
Notons que $H$ peut �tre d�fini par exemple comme le domaine de 
fr�quences pour lesquelles la fonction de transfert du syst�me d'imagerie
n'est pas nulle. Comme nous l'avons d�j� mentionn� (cf~\S I.5),
tout syst�me physique admet une fr�quence de coupure $u_c$ et donc 
$H$ est un domaine born�.

%%%%%%%%%%%%%%%%%%%%%%%%%%%%%%%%%%%%%%%%%%%%%%%%%%%%%%%%%%%
\begin{figure}[h]
\centerline{\epsfxsize=9cm\epsfbox{./figs/decdiag.eps}}
\caption{
Diagramme g�n�ral du probl�me de la d�convolution.}
\label{fig:decdiag}
\end{figure}
%%%%%%%%%%%%%%%%%%%%%%%%%%%%%%%%%%%%%%%%%%%%%%%%%%%%%%%%%%%%%%%%%

Montrons maintenant que 
la d�convolution avec extrapolation spectrale compl�te
est un probl�me mal conditionn�, i.e., que
l'op�rateur inverse $A^{-1}$ n'est pas continu.

Pour d�montrer que $A^{-1}$ n'est pas continu, montrons
tout d'abord que $A$ est un op�rateur de Hilbert-Schmidt (cf~Annexe~\Annexmath).
Il en d�coulera ensuite que $A^{-1}$ n'est pas continu comme l'indique 
la propri�t� suivante (cf.~Annexe~\Annexmath):

{\sl
Soit $A$ un op�rateur de Hilbert-Schmidt d'un espace de Hilbert $E$ 
sur un espace de Hilbert $F$.
Quand $E$ est de dimension infinie et $A$ est injectif, alors 
l'op�rateur $A^{-1}$ de $F$ sur $E$ n'est pas continu (i.e, $A$ n'est 
pas un isomorphisme topologique).
}

Pour montrer que $A$ est un op�rateur de Hilbert-Schmidt, il suffit
de montrer que la trace de $A^* A$ est born�e (cf.~Annexe~\Annexmath).

Montrons tout d'abord que la trace de $A^* A$ est conserv�e par l'op�rateur 
de transform�e de Fourier inverse:
  $$  %%%% 
A  = U^* A_o  \quad \Rightarrow \quad tr A^* A  = tr A_{o}^{*} A_o 
\eqno({\rm II.37})
  $$

En effet:
\begin{eqnarray}
A^* A & = (U^* A_o)^* (U^* A_o) \cr
& = A_{o}^* U U^* A_o \cr
& = A_{o}^{*} A_o 
\qquad (II.38)
\end{eqnarray}

On avait: $A \! = \! U^* h U$. Posons donc: $A_o \! = \! h U$.
Pour toute fonction $\phi$ de $E$, on a:
\begin{eqnarray}
\left ( A_o \phi \right ) (u) 
& = \int_{y \in \Rr^p} h(u) \times \phi (y) \times exp 
\left \{ - 2 {\rm i} \pi u y \right \} \,\, dy \cr
& = \int_{y \in {\Rr^p}} h(u) \times v(y) \times \phi (y) 
\times exp \left \{ - 2 {\rm i} \pi u y \right \} \,\,dy 
\quad (II.39)
\end{eqnarray}

Dans cette expression, nous avons fait appara�tre
$v(y)$ la fonction caract�ristique du domaine $V$, puisque 
$\phi$ est � support sur $V$.

On reconnait la forme d'un {\sl op�rateur int�gral} 
(cf.~Annexe~\Annexmath), i.e:
  $$  %%%% 
(A_o \phi ) (x) \equiv \int_V \Omega_o (x, y) \, \phi (y) \,dy
\eqno({\rm II.40})
  $$

Par identification, on en d�duit l'expression de la fonction 
$\Omega_o$ (cf.~Annexe~\Annexmath):
  $$ %%%%
\Omega_o (u, y) = h(u) \times  v (y) \times exp \left \{ -2 {\rm i} \pi 
u y \right \}  
\eqno({\rm II.41})
  $$

D'apr�s les r�sultats relatifs aux op�rateurs int�graux
de l'annexe~\Annexmath, la trace de $A_0^* A$ s'�crit:
\begin{eqnarray}
tr A_{o}^{*} A_o  & = 2 \times \int \!\!\int
| h(u) \times  v(y) \times  exp \left ( -2 {\rm i} \pi u y  \right ) | ^2 
\,\,dy \,du \cr
 & = 2 \times \int\!\int | h(u)| ^2 \times 
| v(y) |^2 \,\,dy \,du 
\cr
 & = 2 \times \int \mid h(u) \mid^2 \,du \times \int \,\,
 \mid v(y) \mid ^2\,\, dy 
\qquad (II.42)
\end{eqnarray}

Finalement d'apr�s l'�quation (II.38), on obtient l'expression de 
la trace de l'op�rateur $A^* A$:
   $$
tr A^* A  = 2 \times \int_H \mid h(u) \mid^2  du \times 
\int_V \mid v(y) \mid ^2 \,\,dy 
\qquad 2mm (II.43) 
   $$

Les supports $H$ et $V$ �tant born�s, la trace de $A^* A$ est 
donc born�e.
L'op�rateur $A$ est donc de Hilbert-Schmidt. D'apr�s la propri�t�
d�j� mentionn�e au d�but de cette section, 
il en d�coule que $A^{-1}$ n'est pas continu.
Le probl�me de d�convolution avec extrapolation compl�te dans le cas
o� $E \!= \!L^2(V_0)$ est de dimension infinie est donc mal conditionn�.

%%%%%%%%%%%%%%%%%%%%%%%%%%%%%%%%%%%%%%%%%%%%%%%%%%%%%%%%%%%%%%%%%%%%%%
\def\comment{%%%%%
\bigskip
\noindent
{\sl Cas de dimension finie}

\penalty 10000
\medskip
Le passage � la dimension finie, i.e., dans le cas o� les supports
$V_O$ et $H$  sont ``�chantillonn�s'' spatialement (et o� donc
l'espace $E \!= \!L^2(V_0)$ n'est plus de dimension infinie),
r�gularise quelque peu le probl�me.
Les m�thodes it�ratives convergent mais avec un nombre 
d'it�rations tr�s importants.
En fait la stabilit� du processus est quasiment impossible � 
contr�ler.
}%%%%%%%%%%%%%%%%%%%%%%%%%%%%%%%%%%%%%%%%%%%%%%%%%%%%%%%%%%%%%%%%%

%%%%%%%%%%%%%%%%%%%%%%%%%%%%%%%%%%%%%%%%%%%%%%%%%%%%%%%%%%%
\subsection{Limitation de la r�solution: fonction de lissage}
 
Nous venons de voir que la d�convolution avec extrapolation
spectrale compl�te conduisait � un probl�me mal
conditionn�. Il faut donc reformuler le probl�me en termes 
d'extrapolation (ou interpolation) spectrale partielle dans une 
certaine r�gion spectrale $W$ de $\Rr^p$. Ceci revient en fait � 
imposer une contrainte sur le comportement du spectre
de l'objet restaur� pour les hautes fr�quences, i.e.,
une {\sl limitation en r�solution}.

On est ainsi conduit � d�finir une version liss�e $\phi_s$ de l'objet 
d'origine $\phi_o$ � un degr� de r�solution plus faible, i.e, 
$\phi_s$ sera plus pauvre en hautes fr�quences spatiales que 
$\phi_o$, et $ |\widehat \phi_s (u) |$ sera ``petit'' au sens des 
moindres carr�s en dehors d'une r�gion born�e $H_r$ de l'espace 
des fr�quences. 
On d�finira $H_r$ dans $\Rr^p$ comme la r�gion "r�gularisant"
de la meilleure fa�on possible le support $H$ de la fonction de 
transfert. $H_r$ sera appel�e {\sl ouverture synth�tique}.

L'objectif sera donc d�sormais la restitution d'une version
limit�e en r�solution de l'objet initial $\phi_o$:
  $$  %%%% 
\phi_s  =  S \phi_o, 
  $$
o� $S$ est un op�rateur de lissage irr�versible, de $E_o$ dans 
l'espace $E$ de {\sl reconstruction objet}.
$S$ est d�termin� en fonction du syst�me d'imagerie. 

Lannes {\it et al.}, (1987) a d�fini 
cet op�rateur de la fa�on suivante:

$S$ est un op�rateur de convolution par une fonction $s$
(i.e., $ S \! = \! U^* s U$):
  $$ %%%%%
\phi_s  =  s * \phi_o 
\eqno({\rm II.44})
  $$ 

La {\sl fonction de lissage} $ \widehat s (u)$ est paire et � valeurs r�elles, 
et satisfait aux trois propri�t�s suivantes:

$\bullet$ l'�nergie de $ \widehat s$ est concentr�e dans $H_r$, i.e, 
$ \mid \widehat s (u) \mid $ est petit 
au sens des moindres carr�s en dehors de $H_r$.
Posons :
  $$ %%%% 
\chi^2 ={ 1 \over \parallel \widehat s \parallel ^2} 
\int_{H_r} \mid \widehat s(u) \mid ^2 \, du 
\eqno({\rm II.45})
  $$
o� $\chi^2$ repr�sente la fraction d'�nergie 
de $\widehat s$ dans $H_r$.
On imposera comme condition que 
$\chi^2$ soit proche de~1. 

$\bullet$ $\widehat s (u)$ est normalis�e conform�ment � la relation 
$s(0) \! = \! \int \widehat s(u) \,du \! = \! 1$
ce qui implique
$ \int  \widehat \phi_s (u)\,du \! = \! \int \widehat \phi_o (u)\,du $,
soit encore (par l'application du th�or�me de Parseval)
$ \int \phi_s (x)\,dx \! = \! \int \phi_o (x)\,dx $
(conservation de la photom�trie).

$\bullet$ Le support $D_r$ (dans l'espace direct) de $s$ est aussi 
petit que possible, i.e, la r�solution de $\phi_s$ doit �tre la 
meilleure possible. La taille du support $D_r$ de $s$ est ajust�e
en fonction de la r�solution que l'on souhaite obtenir pour $\phi_s$.

\bigskip
On peut maintenant d�finir plus pr�cis�ment l'espace $E$ de 
reconstruction objet: il s'agit de $L^2 (V)$ o� $ V \! = \! V_o + D_r$
($V_o$ agrandi par la fonction d'�talement $s$ dont le support 
est $D_r$).
En effet le support de $\phi_s$ est contenu dans la ``somme des 
supports'' de $\phi_o$ et de $s$ (voir Fig.~\ref{fig:decregul}).

Lannes {\it el al}., 1987, ont montr� qu'un choix judicieux des
supports $D_r$ et $H_r$ de cet op�rateur $S$ r�gularisait le probl�me
de la d�convolution tout en autorisant une certaine extrapolation
ou interpolation dans le domaine de Fourier, i.e., un gain 
en r�solution. Nous reviendrons en \S II.3.f sur le crit�re
quantitatif introduit par ces auteurs.

%%%%%%%%%%%%%%%%%%%%%%%%%%%%%%%%%%%%%%%%%%%%%%%%%%%%%%%%%%%
\subsection{ Interpolation des ``trous fr�quentiels''}

La mesure $\hat{\phi}(u)$ du spectre de l'image
fournie par un syst�me d'imagerie
n'est g�n�ralement pas de qualit� uniforme sur toutes les fr�quences.
Le rapport signal sur bruit (SNR) est souvent meilleur pour les 
basses fr�quences, diminue vers les hautes fr�quences, puis
s'annule au-del� de la fr�quence de coupure du syst�me d'imagerie.
On peut donc imposer un seuil $\alpha_t$ en de�� duquel le 
rapport SNR est trop faible pour que les donn�es
soient consid�r�es comme fiables. 
Dans les zones concern�es,
appel�es {\sl trous fr�quentiels}, on peut soit rejeter en bloc 
les donn�es et les consid�rer comme nulles, soit proc�der � 
une interpolation, en utilisant au besoin des contraintes
sur l'objet � reconstruire.

Dans le premier cas, on obtient une approximation grossi�re 
$\hat{\Phi}_t(u)$ du spectre de l'objet en divisant 
par la fonction de transfert
le spectre de l'image, tronqu� par 
$\hat{s}(u)$ sur le domaine $H_r$, l� o� le rapport SNR est 
sup�rieur � $\alpha_t$:
  $$  %%%% 
\hat{\Phi}_t (u) = 
\hat{s}_t(u) \hat{\psi}_i (u)
 \qquad {\rm avec} \quad
\hat{s}_t(u) = 
\left\{ \!\!
\matrix{
  & \hat{s}(u) / \hat{h}(u) 
  & {\rm si} \quad SNR(u) \geq \alpha_t ;\cr
  & 0 
  & {\rm sinon.} \cr
        }
\right.
\eqno({\rm II.46})
  $$

Dans le second cas, l'interpolation peut se faire tout
simplement en minimisant la fonctionnelle:
  $$  %%%% 
q(\phi) = \| \, g \left( \hat{\Phi}_t - \hat{\phi} \right) 
\|^2
\eqno({\rm II.47})
  $$
sur l'espace $E$ de reconstruction objet.
La fonction poids $g(u)$ est d�finie 
de fa�on � ne pas tenir compte des donn�es de mauvaise qualit�:
  $$  %%%% 
g(u) = 
\left\{
\matrix{
  & 1  & {\rm si} \quad SNR(u) \geq \alpha_t ;\cr
  & 0  & {\rm sinon.} \cr
        }
\right.
\eqno({\rm II.48})
  $$

{\sl L'interpolation dans le domaine de Fourier n'est donc pas 
explicite.} La solution sera d�finie par la
fonction appartenant � $E$ dont le spectre minimise la distance 
avec $\hat{\Phi}_t$, en tenant compte des poids $g(u)$.
La fonction $w(u) = 1 - g^2(u)$ est donc
caract�ristique du domaine d'interpolation $W$ dans l'espace de 
Fourier. 

Notons de plus que $g(u)$ vaut 1 en dehors de $H_r$, car il ne
doit pas y avoir d'interpolation; on doit s'efforcer d'avoir des valeurs 
nulles pour le spectre de la fonction solution en dehors de 
$H_r$. On peut aussi choisir une fonction $g$ qui ne soit pas 
binaire, et qui affecte un poids proportionnel au rapport SNR 
lorsque celui-ci est sup�rieur � $\alpha_t$.

\bigskip
Rappelons que $U$ et $U^{*}$ sont respectivement les op�rateurs 
de transform�e de Fourier, et l'op�rateur inverse.
Reprenons l'expression II.47, en y substituant
$\hat{\phi} = U \phi $ et $\hat{\Phi}_t = \hat{s}_t U \psi_i$. 
La fonctionnelle � minimiser s'�crit alors:
  $$  %%%% 
q(\phi) = \| g \left( \hat{s}_t U \psi_i - U \phi \right) \|^2
\eqno({\rm II.49})
  $$

\smallskip
En utilisant le th�or�me de Parseval (conservation du produit 
scalaire par la transform�e de Fourier):
  $$  %%%% 
q(\phi) = \| U^{*} g \left( \hat{s}_t U \psi_i - U \phi \right) \|^2
\eqno({\rm II.50})
  $$

\smallskip
On reconnait alors un probl�me-type des moindres carr�s 
(Cf.~annexe~\Annexmcarres): 
  $$  %%%% 
q(\phi) = \| \psi_i - A \phi \|^2 
\quad \hbox{avec} \quad
\Psi_i =  U^{*} g \hat{s}_t U \psi_i
\quad \hbox{et} \quad
A \phi =  U^{*} g U \phi
\eqno({\rm II.51})
  $$

La fonction recherch�e $\Phi$ est donc solution de l'�quation normale
$A^{*}A \phi = A^{*} \psi_i$.

D�terminons maintenant l'op�rateur adjoint $A^{*}$. Soit $F=AE$
l'espace image des fonctions de $E$. Pour tout $\psi$ de $F$, on 
a: 
  $$  %%%% 
(A \phi | \psi)_F = (\phi | A^{*} \psi)_E 
\eqno({\rm II.52})
  $$
Donc:
  $$  %%%% 
(A \phi | \psi)_F = 
(U^{*} g U \phi | \psi)_F 
=(\phi | v U^{*} g U \psi)_E 
\eqno({\rm II.53})
  $$

Nous avons introduit ici $v(x)$, la fonction caract�ristique du support 
$V$ de l'objet, car nous imposons une contrainte de support dans 
l'espace direct.  On en d�duit:
  $$  %%%% 
A^{*} = v U^{*} g U 
\qquad {\rm et}
\qquad
A^{*}A = v U^{*} g^2 U 
\eqno({\rm II.54})
  $$

L'�quation normale du probl�me de d�convolution avec interpolation
partielle s'�crit donc:
  $$
v U^{*} g^2 U \phi = v U^{*} g U \psi_i
\qquad (II.55)
  $$

%%%%%%%%%%%%%%%%%%%%%%%%%%%%%%%%%%%%%%%%%%%%%%%%%%%%%%%%%%%
\begin{figure}[h]
\centerline{\epsfxsize=9cm\epsfbox{./figs/decregul.eps}}
\caption{
Illustration unidimensionnelle du principe de r�gularisation: on introduit
une fonction de lissage $\widehat s$ ``confin�e'' sur $H_r$ qui peut
dans certains cas �tre plus �tendu que 
le support $H$ de la fonction de transfert du syst�me d'imagerie
(extrapolation partielle).}
\label{fig:decregul}
\end{figure}
%%%%%%%%%%%%%%%%%%%%%%%%%%%%%%%%%%%%%%%%%%%%%%%%%%%%%%%%%%%%%%%%%

%%%%%%%%%%%%%%%%%%%%%%%%%%%%%%%%%%%%%%%%%%%%%%%%%%%%%%%%%%%
\subsection{ Stabilit� de la d�convolution}

Examinons maintenant les conditions de stabilit�, qui vont �tre
d�termin�es par les valeurs propres de la matrice $A^{*}A$ 
(cf.~annexe~\Annexmcarres).

En utilisant la fonction  $w = 1 - g^2$, il vient:
  $$  %%%% 
A^{*}A \phi = v U^{*} (1 - w) U \phi = v \phi - v U^{*} w U \phi
\eqno({\rm II.56})
  $$

En tenant compte des contraintes de support: $v \phi = \phi $.
Donc:
  $$  %%%% 
A^{*}A = I - B
\qquad {\rm avec}
\quad
B = v U^{*} w U 
\eqno({\rm II.57})
  $$

L'op�rateur $B$ est caract�ris� par deux troncatures successives:
sur $W$ dans l'espace de Fourier, puis sur $V$ dans l'espace 
direct. On montre alors que la stabilit� du processus de 
reconstruction est li�e au param�tre $\eta= \tau \nu$,  
o� $\tau^2$ et $\nu^2$ sont les surfaces respectivement de $V$ et $W$:
$\tau^2 = \int_{V} dx$
et $\nu^2 = \int_{W} du$.
Ce param�tre $\eta$ caract�rise la quantit� d'interpolation qui 
est effectu�e � la fois dans l'espace r�el et dans l'espace de 
Fourier.

%%%%%%%%%%%%%%%%%%%%%%%%%%%%%%%%%%%%%%%%%%%%%%%%%%%%%%%%%%%%%%%%%
\def\comment{%%%%%%%%%%%%%%%%%%%%%%%%%%%%%%%%%%%%%%%%%%%%%%%%%%
Le param�tre-cl� est alors le {\sl param�tre d'interpolation}
$\eta$ qui est le produit des racines carr�es des ``aires'' de $H_r$ et $D_r$:

  $$ %%%%%
\eta = 
\left( \int_{H_r} \, du \right)^{1/2} 
\times \left( \int_{D_r} \, dx \right)^{1/2}
\eqno({\rm II.58})
  $$ 
 
Ce param�tre permet ainsi de quantifier la ``r�gularisation du probl�me,''
en limitant la r�solution de la fonction objectif � restaurer
(d�pendant de l'ouverture synth�tique $H_r$ et 
de l'ellipso�de de r�solution $D_r$)
}%%%%%%%%%%%%%%%%%%%%%%%%%%%%%%%%%%%%%%%%%%%%%%%%%%%%%%%%%%%%%%%%%

{\sl Conclusion:} La stabilit� du processus de d�convolution 
d�pend du nombre de conditionnement de l'op�rateur $A^{*}A$.
Si le probl�me est mal conditionn�, on doit r�duire la taille de 
$V$ ou celle de $W$, 
ce qui dans ce dernier cas limite la r�solution que l'on peut 
atteindre. Pour �viter des calculs inutiles, il est judicieux de
calculer tout d'abord la valeur de $\eta$, qui est �troitement li� au
nombre de conditionnement. Lorsque ce nombre prend des valeurs 
raisonnables (inf�rieur � quelques unit�s), on peut engager le 
processus complet de minimisation par moindres carr�s. 

\bigskip
\noindent
{\sl Conditions de stabilit� de reconstruction}

\penalty 10000
\medskip
On supposera que le spectre de $A^* A$ admet un et un seul point 
d'accumulation $\mu''$ et que de plus $\mu'' \not \! = \! 0$.
Autrement dit, $A^* A$ �tant positif, ceci signifie que:

-- La plus petite valeur propre $\mu$ de $A^* A$ est strictement 
positive:
  $$ %%%%  
\displaystyle \mu = \min_{\parallel \Phi \parallel = 1} 
{\parallel A \Phi \parallel ^2  > 0} 
\eqno({\rm II.59})
  $$
-- La suite des valeurs propres $(\mu_j)_{j \geq 1}$ est telle 
que:
  $$ %%%% 
\displaystyle \lim_{j \rightarrow \infty } {\mu_j} = \mu'' 
\eqno({\rm II.60})
  $$

La premi�re condition signifie que $F$ est ferm� car $A$ est un 
isomorphisme de $E$ dans $F$ avec 
$ \parallel A^{- 1} \parallel \! = \! 1 / \sqrt \mu$.
La connaissance de $\mu$ permet d'estimer  la fiabilit� du 
processus de reconstruction consid�r�. 
  $$ %%%%
{ \parallel \delta \Phi \parallel \over \parallel \Phi 
\parallel } \leq  \sqrt{{\mu' \over \mu}} { \parallel \delta \Psi 
\parallel \over \parallel \Psi \parallel }
\eqno({\rm II.61})
  $$
o� $ \displaystyle \mu' \! = \! sup_{\parallel \Phi \parallel \! = \! 1 } 
{\parallel A \Phi  \parallel ^2} \! =  \parallel A \parallel ^2 $
et $ C_a \! = \! \sqrt {\mu' / \mu''} $  est le nombre de 
conditionnement.
Le processus de reconstruction sera d'autant plus fiable que 
$C_a$ sera plus proche de~1. De plus, on voit que la condition 
$\mu \! \not = \! 0$ asssure l'existence de $C_a$.

Pour l'interpr�tation de la deuxi�me condition, on consid�re 
l'�quation  aux valeurs propres $ A^* A \Phi \! = \! M \Phi$,
le param�tre $M$ prend toutes les valeurs de la suite 
$(\mu_j)_{j \geq 1}$, or si $\mu''$ est point d'accumulation du 
spectre de $A^* A$, cela signifie que $M$ prendra un tr�s grand 
nombre de fois une valeur proche de~$\mu''$. On peut alors 
remplacer l'�quation pr�c�dente par: $ \mu'' \Phi - A^* A $ est 
tr�s petit. Ce qui signifie que  l'op�rateur $ B \! = \! \mu'' I - A^* 
A$ ne provoque qu'une petite perturbation des fonctions sur 
lesquelles il op�re.
Pour que le probl�me soit bien conditionn�, il faut que $A^* A$ 
soit une petite perturbation de l'identit� et donc qu'il ait une 
action relativement faible sur les fonctions sur lesquelles il va 
op�rer.

%%%%%%%%%%%%%%%%%%%%%%%%%%%%%%%%%%%%%%%%%%%%%%%%%%%%%%%%%%%
\subsection{ Estimation des erreurs}

Par rapport � la plupart des autres m�thodes de d�convolution,
cette minimisation par moindres carr�s pr�sente l'avantage
de pouvoir estimer les erreurs de reconstruction. 

Il est bon de v�rifier que le vecteur propre associ� � la 
plus petite des valeurs propres $\mu$ de $A^{*} A$ ne domine pas la 
solution, ce qui indiquerait que les structures qui lui sont 
associ�es sont des artefacts.

Comme dans tout
processus de minimisation par moindres carr�s (Cf. \S II.1), 
la solution $\Phi$ de l'�quation normale
est la projection orthogonale de $\Psi_i$ sur $F=AE$.
Les erreurs de reconstruction sont d�finies par 
$\Delta \Phi = \| \Phi - \phi_s \| $  et 
$\Delta \Psi = \| \Psi - \Psi_s \|$
dans les espaces objet et image respectivement
(avec $A \phi_s = \Psi_s$). On a donc:
  $$  %%%% 
\| \Psi_i - \Psi_s \|^2 
= \| \Psi_i - \Psi \|^2 + \| \Psi - \Psi_s \|^2
= \| \Psi_i - \Psi \|^2 + \| \Delta \Psi \|^2
\eqno({\rm II.62})
  $$

%%%%%%%%%%%%%%%%%%%%%%%%%%%%%%%%%%%%%%%%%%%%%%%%%%%%%%%%%%%
\begin{figure}[h]
\centerline{\epsfxsize=9cm\epsfbox{./figs/decerror.eps}}
\caption{
Principe de l'estimation des erreurs.}
\label{fig:decerror}
\end{figure}
%%%%%%%%%%%%%%%%%%%%%%%%%%%%%%%%%%%%%%%%%%%%%%%%%%%%%%%%%%%%%%%%%

Un majorant $\epsilon$ de la distance $\| \Psi_i - \Psi_s \|$, entre la
fonction image mesur�e $\Psi_i$ et 
l'image th�orique $\Psi_s$ de l'objet liss� $\phi_s$ par l'op�rateur 
imageur $A$, peut �tre calcul� par une estimation des erreurs
d'entr�e (par le rapport SNR). On en d�duit:
$\| \Delta \Psi \| \leq \epsilon$

Or $\Delta \Phi = A^{-1} \Delta \Psi$, donc:
  $$  %%%% 
\| \Delta \Phi \| 
\leq \| A^{-1} \| \; \|\Delta \Psi \|
= { 1 \over \sqrt{\mu_{min}} } \; \|\Delta \Psi \|
\eqno({\rm II.63})
  $$
o� $\mu_{min}$ est la plus petite des valeurs propres de 
$A^{*} A$ (et $\sqrt{\mu_{min}}$ est alors la plus petite des 
valeurs singuli�res de l'op�rateur $A$). 
On en d�duit:
  $$  %%%% 
\| \Delta \Phi \| 
\leq 
{ \epsilon \over \sqrt{\mu_{min}} } 
\eqno({\rm II.64})
  $$

On peut bien s�r faire une analyse plus fine des erreurs avec 
une d�composition compl�te en valeurs propres de l'op�rateur
$A^{*}A$ pour d�terminer si les erreurs sont bien r�parties sur 
tout le spectre des vecteurs propres.
Il peut �tre utile de v�rifier que le vecteur propre associ� � la 
plus petite des valeurs propres $\mu_{min}$ de $A^{*} A$ ne domine pas la 
solution, ce qui indiquerait que les structures qui lui sont 
associ�es sont des artefacts.
 

\def\fullref#1{{\sl #1}}
%%%%%%%%%%%%%%%%%%%%%%%%%%%%%%%%%%%%%%%%%%%%%%%%%%%%%%%%%%%%%%%%%%%%%%
\centerline{\large \bf Bibliographie}

\bigskip
\parindent=0pt
% \bibitem[Lannes at~al.(1987a)]{lannes87a}
\fullref{Stabilized reconstruction in signal and image processing;
Part~I: Partial deconvolution and spectral extrapolation with limited field}

Lannes, A., Roques, S., Casanove, M.J., 1987a, J. Mod. Optics, 34, 161--226

\medskip
% \bibitem[Lannes at~al.(1987b)]{lannes87b}
\fullref{Stabilized reconstruction in signal and image processing;
Part~II: Iterative reconstruction with and without constraint.
Interactive implementation,}

Lannes, A., Roques, S., Casanove, M.J., 1987b, J. Mod. Optics, 34, 321--370

\medskip
% \bibitem[Lannes(1994)]{lannes94}
\fullref{Fourier interpolation and reconstruction via Shannon-type techniques.
I. Regularization principle}

Lannes, A., Anterrieu, E., Bouyoucef, K., 1994, J. Mod. Optics, 41, 1537--1574

\medskip
% \bibitem[Lannes(1996)]{lannes96}
\fullref{Fourier interpolation and reconstruction via Shannon-type techniques.
II. Technical developments and applications}

Lannes, A., Anterrieu, E., Bouyoucef, K., 1996, J. Mod. Optics, 43, 105--138

\medskip
\fullref{Probl�mes inverses en traitement d'image. R�gularisation et r�solution en imagerie bidimensionnelle,}

Roques, S., Th�se d'Etat, 1987, Univ. Paul Sabatier, Toulouse





%%%%%%%%%%%%%%%%%%%%%%%%%%%%%%%%%%%%%%%%%%%%%%%%%%%%%%%%%%%
\end{document}

%%%% Local Variables:
%%%% mode: latex
%%%% TeX-master: t
%%%% TeX-display-help: yes
%%%% End:
