%% Simulation No3
%%%%%%%%%%%%%%%%%%%%%%%%%%%%%%%%%%%%%%%%%%%%%%%%%%%%%%%%%%%
\begin{figure}[h]
\centerline{\epsfxsize=7.5cm\epsfbox{s5_yy0.ps}
\epsfxsize=7.5cm\epsfbox{s5_yyb.ps}}
\vskip 1mm
\centerline{\epsfxsize=7.5cm\epsfbox{s5_wiener_1.ps}
\epsfxsize=7.5cm\epsfbox{s5_tikho_10.ps}}
\vskip 1mm
\centerline{\epsfxsize=7.5cm\epsfbox{s5_ggauss_10.ps}
\epsfxsize=7.5cm\epsfbox{s5_mem_1e-5.ps}}
\caption{Simulation \#5. 
De gauche � droite et de haut en bas: 
objet original, 
image filtr�e et bruit� avec SNR=13 dB,
d�convolution par filtre de Wiener ($\alpha$=1, erreur rms=0.0100),
avec r�gularisation de Tikhonov ($\alpha$=10., it=144, rms=0.0097), 
avec r�gul. g�n�ralis�e de Gauss (p=1.1, $\alpha$=10, it=1000, rms=0.0103), 
avec maximum d'entropie ($\alpha$=1e-5, it=59, rms=0.0099), 
Pour la d�convolution, le crit�re d'arr�t a �t� fix� � $10^{-6}$.}
\label{fig:s5-simu}
\end{figure}

%%%%%%%%%%%%%%%%%%%%%%%%%%%%%%%%%%%%%%%%%%%%%%%%%%%%%%%%%%%
\begin{figure}[h]
\centerline{\epsfxsize=7.5cm\epsfbox{s5_sqrt_10_01.ps}
\epsfxsize=7.5cm\epsfbox{s5_sqrt_10_1.ps}}
\vskip 1mm
\centerline{\epsfxsize=7.5cm\epsfbox{s5_sqrt_10_01_lin.ps}
\epsfxsize=7.5cm\epsfbox{s5_sqrt_10_1_lin.ps}}
\caption{Simulation \#5.
D�convolution avec r�gularisation convexe $\sqrt{s^2+x^2}$
avec $s$=0.1, $\alpha$=10., it=1000., rms=0.0104(� gauche)
et $s$=1., $\alpha$=10., it=792, rms=0.0101 (� droite)
(tol�rance = $10^{-6}$). En haut, �chelle logarithmique,
en bas, �chelle lin�aire.}
\end{figure}

%%%%%%%%%%%%%%%%%%%%%%%%%%%%%%%%%%%%%%%%%%%%%%%%%%%%%%%%%%%
\def\comments{
\begin{figure}[h]
\centerline{\epsfxsize=7.5cm\epsfbox{s4_gmark_1.ps}}
\caption{Simulation \#5.
D�convolution avec r�gularisation Gauss-Markov 
avec $\alpha$=1., it=536, rms=0.0519 (� gauche)
(tol�rance = $10^{-6}$).}
\end{figure}
}
